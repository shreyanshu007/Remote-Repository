\documentclass[journal]{IEEEtran}
\usepackage{amsthm}
\usepackage{url}
\usepackage{graphicx}

\begin{document}

\title{Digital Survey Portal}

\author{

Shreyanshu Shekhar\\
2016csb1060@iitrpr.ac.in

}


\maketitle
\begin{abstract}
This is and idea to increase the interaction between public and government. With the help of this portal people can communicate with the government and directly express their problems. Government can use this to take suggestions and conduct survey. This will help them in decision making regarding funds and policies.

\end{abstract}

%\begin{IEEEkeywords}
%Audio, task1, task2, task3, application
%\end{IEEEkeywords}

\section{Introduction}
These days our country's main focus is towards digitization so this project can be step towards it.

Under different ministry government has many funds but due to lack of current data, funds are not used properly. But if in any way government know that certain village needs some work to be done in education sector or certain city need some awareness in health sector then government can use their funds and potential in that specific area. This will help the government to use their resources in the best way. 

So there is proposal of a web portal which will help both government and public. If government has funds and interested in making policies then they can decide that which area needs which type of policies. They can conduct a survey in different sectors like Education, Health and Agriculture. 

Every sector has its own Problems. People can respond to the survey. The results of survey will highlight which sector needs what kind of care. This will help the government to priorities the issues. 

\section{Implementation}
These days government does many surveys for collecting different data. For that they require lots of man power. For example survey for India's population government officers get involved going to each person individually, but it doesn't seems a good idea these days.

Due to digitization these things becomes very easy. This portal will be able to provide all the options for these types of things.

There are sectors from where data can be collected and using those data government can act properly. These data will provide the ground details like which city or village requires which type of care. 

Issues can be:

\begin{itemize}
\item Health
\begin{itemize}
    \item Child health
    \item women health
    \item Spreading disease
    \item Drinking water
    \item Sanitation problems
\end{itemize}

\item Education
\begin{itemize}
    \item Child education
    \item Literacy
    \item Primary education
    \item Availability of books
\end{itemize}

\item Agriculture
\begin{itemize}
    \item Fertility of soil
    \item Irrigation
    \item Market
    \item Capital
\end{itemize}
\end{itemize}

\section{Technology stack}
We need to create databases to store the responses given by peoples.

We would be using following technologies for our project:
\begin{itemize}
\item  HTML, CSS and JavaScript for front end programming 
\item  PHP for back-end programming
\item  The database we would use will be MySQL with server Apache 
\end{itemize}


\section{Use cases}
Main users will be government and public. 

Use cases of this project will be:
\begin{itemize}
\item  Any government department can conduct customize survey for their respective department.
\item  Public can respond to the survey using their aadhaar number.
\item  Public can also give suggestions regarding any department.
\item Those suggestions can be managed by respective department officials.
\end{itemize}
  
    

\end{document}
\grid
\grid
\grid
