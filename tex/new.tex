% This is "sig-alternate.tex" V1.9 April 2009
% This file should be compiled with V2.4 of "sig-alternate.cls" April 2009
%
% This example file demonstrates the use of the 'sig-alternate.cls'
% V2.4 LaTeX2e document class file. It is for those submitting
% articles to ACM Conference Proceedings WHO DO NOT WISH TO
% STRICTLY ADHERE TO THE SIGS (PUBS-BOARD-ENDORSED) STYLE.
% The 'sig-alternate.cls' file will produce a similar-looking,
% albeit, 'tighter' paper resulting in, invariably, fewer pages.
%
% ----------------------------------------------------------------------------------------------------------------



% This .tex file (and associated .cls V2.4) produces:
%       1) The Permission Statement
%       2) The Conference (location) Info information
%       3) The Copyright Line with ACM data
%       4) NO page numbers
%
% as against the acm_proc_article-sp.cls file which
% DOES NOT produce 1) thru' 3) above.
%
% Using 'sig-alternate.cls' you have control, however, from within
% the source .tex file, over both the CopyrightYear
% (defaulted to 200X) and the ACM Copyright Data
% (defaulted to X-XXXXX-XX-X/XX/XX).
% e.g.
% \CopyrightYear{2007} will cause 2007 to appear in the copyright line.
% \crdata{0-12345-67-8/90/12} will cause 0-12345-67-8/90/12 to appear in the copyright line.
%
% ---------------------------------------------------------------------------------------------------------------
% This .tex source is an example which *does* use
% the .bib file (from which the .bbl file % is produced).
% REMEMBER HOWEVER: After having produced the .bbl file,
% and prior to final submission, you *NEED* to 'insert'
% your .bbl file into your source .tex file so as to provide
% ONE 'self-contained' source file.
%
% ================= IF YOU HAVE QUESTIONS =======================
% Questions regarding the SIGS styles, SIGS policies and
% procedures, Conferences etc. should be sent to
% Adrienne Griscti (griscti@acm.org)
%
% Technical questions _only_ to
% Gerald Murray (murray@hq.acm.org)
% ===============================================================
%
% For tracking purposes - this is V1.9 - April 2009

\documentclass{sig-alternate}
\usepackage{blkarray}

\usepackage{array}
\newcolumntype{L}[1]{>{\raggedright\let\newline\\\arraybackslash\hspace{0pt}}m{#1}}


\begin{document}
%
% --- Author Metadata here ---

\conferenceinfo{MM'12}{ October 29-November 2, 2012, Nara, Japan.}
%\CopyrightYear{2007} % Allows default copyright year (20XX) to be over-ridden - IF NEED BE.
%\crdata{0-12345-67-8/90/01}  % Allows default copyright data (0-89791-88-6/97/05) to be over-ridden - IF NEED BE.
% --- End of Author Metadata ---



\title{Digital Survey Portal}


\numberofauthors{8} %  in this sample file, there are a *total*
% of EIGHT authors. SIX appear on the 'first-page' (for formatting
% reasons) and the remaining two appear in the \additionalauthors section.
%
\author{
\alignauthor Shreyanshu Shekhar\\
       \affaddr{Dept. of Computer Science}\\
       \affaddr{IIT  Ropar}\\
       \email{2016csb1060@iitrpr.ac.in}
% 2nd. author
\alignauthor Akshat Rathor\\
       \affaddr{Dept. of Computer Science}\\
       \affaddr{IIT  Ropar}\\
       \email{2016csb1030@iitrpr.ac.in}
\and  
% 3rd. author
\alignauthor Sameer Arora\\
       \affaddr{Dept. of Computer Science}\\
       \affaddr{IIT  Ropar}\\
       \email{2016csb1058@iitrpr.ac.in}
% 4th. author
\alignauthor Sahil Kumar\\
       \affaddr{Dept. of Computer Science}\\
       \affaddr{IIT  Ropar}\\
       \email{2016csb1056@iitrpr.ac.in}
}


% There's nothing stopping you putting the seventh, eighth, etc.
% author on the opening page (as the 'third row') but we ask,
% for aesthetic reasons that you place these 'additional authors'
% in the \additional authors block, viz.

% Just remember to make sure that the TOTAL number of authors
% is the number that will appear on the first page PLUS the
% number that will appear in the \additionalauthors section.

\maketitle
\begin{abstract}
This is and idea to increase the interaction between
public and government. With the help of this portal people can
communicate with the government and directly express their
problems. Government can use this to take suggestions and
conduct survey. This will help them in decision making regarding
funds and policies
\end{abstract}



% A category with the (minimum) three required fields
%\category{H.4}{Information Systems Applications}{Miscellaneous}
%A category including the fourth, optional field follows...
%\category{D.2.8}{Software Engineering}{Metrics}[complexity measures, performance measures]

%\terms{Theory}

%\keywords{ACM proceedings, \LaTeX, text tagging}

\section{Introduction}
These days our country's main focus is towards digitization so this project can be step towards it.

Under different ministry government has many funds but due to lack of current data, funds are not used properly. But if in any way government know that certain village needs some work to be done in education sector or certain city need some awareness in health sector then government can use their funds and potential in that specific area. This will help the government to use their resources in the best way. 

So there is proposal of a web portal which will help both government and public. If government has funds and interested in making policies then they can decide that which area needs which type of policies. They can conduct a survey in different sectors like Education, Health and Agriculture. 

Every sector has its own Problems. People can respond to the survey. The results of survey will highlight which sector needs what kind of care. This will help the government to priorities the issues. 

\section{Implementation}
These days government does many surveys for collecting different data. For that they require lots of man power. For example survey for India's population government officers get involved going to each person individually, but it doesn't seems a good idea these days.

Due to digitization these things becomes very easy. This portal will be able to provide all the options for these types of things.

There are sectors from where data can be collected and using those data government can act properly. These data will provide the ground details like which city or village requires which type of care. 

Issues can be:

\begin{itemize}
\item Health
\begin{itemize}
    \item Child health
    \item women health
    \item Spreading disease
    \item Drinking water
    \item Sanitation problems
\end{itemize}

\item Education
\begin{itemize}
    \item Child education
    \item Literacy
    \item Primary education
    \item Availability of books
\end{itemize}

\item Agriculture
\begin{itemize}
    \item Fertility of soil
    \item Irrigation
    \item Market
    \item Capital
\end{itemize}
\end{itemize}

\section{Technology stack}
We need to create databases to store the responses given by peoples.

We would be using following technologies for our project:
\begin{itemize}
\item  HTML, CSS and JavaScript for front end programming 
\item  PHP for back-end programming
\item  The database we would use will be MySQL with server Apache 
\end{itemize}


\section{Use cases}
Main users will be government and public. 

Use cases of this project will be:
\begin{itemize}
\item  Any government department can conduct customize survey for their respective department.
\item  Public can respond to the survey using their aadhaar number.
\item  Public can also give suggestions regarding any department.
\item Those suggestions can be managed by respective department officials.
\end{itemize}



\end{document}



